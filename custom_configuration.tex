%% Information for inside title page


\authorsLastname{Sanchez Torres}
\authorsFirstname{Andres Alam}
\email{aast2@kth.se}
\kthid{u100001}
% As per email from KTH Biblioteket on 2021-06-28 students cannot have an OrCiD reported for their degree project
\authorsSchool{\schoolAcronym{EECS}}
%If the student is not in Stockholm, Sweden, add that information here
% This information will be used when generating the acknowledgements signature.
%\authorCity{A City}
%\authorCountry{A Country}
% pass into \authorCityCountryDate{} the month and year for the acknowledgment
% If there is a second author and place, the month, and year are the same, the specify the month and year for the first author:
\authorCityCountryDate{\MONTH\enspace\the\year}
% if there is a second author and the place is different, then say:
%\authorCityCountryDate{}

\supervisorAsLastname{Öktem}
\supervisorAsFirstname{Ozan}
\supervisorAsEmail{ozan@kth.se}
% If the supervisor is from within KTH add their KTHID, School and Department info
\supervisorAsKTHID{u100003}
\supervisorAsSchool{\schoolAcronym{SCI}}
\supervisorAsDepartment{Mathematics}
% other for a supervisor outside of KTH add their organization info
%\supervisorAsOrganization{Timbuktu University, Department of Pseudoscience}

% %If there is a second supervisor add them here:
% \supervisorBsLastname{Supervisor}
% \supervisorBsFirstname{Another Busy}
% \supervisorBsEmail{sb@kth.se}
% % If the supervisor is from within KTH add their KTHID, School and Department info
% \supervisorBsKTHID{u100003}
% \supervisorBsSchool{\schoolAcronym{ABE}}
% \supervisorBsDepartment{Architecture}
% other for a supervisor outside of KTH add their organization info
%\supervisorBsOrganization{Timbuktu University, Department of Pseudoscience}

% %If there is a third supervisor add them here:
% \supervisorCsLastname{Supervisor}
% \supervisorCsFirstname{Third Busy}
% \supervisorCsEmail{sc@tu.va}
% % If the supervisor is from within KTH add their KTHID, School and Department info
% %\supervisorCsKTHID{u100004}
% %\supervisorCsSchool{\schoolAcronym{ABE}}
% %\supervisorCsDepartment{Public Buildings}
% % other for a supervisor outside of KTH add their organization info
% \supervisorCsOrganization{Timbuktu University, Department of Pseudoscience}

\examinersLastname{Öktem}
\examinersFirstname{Ozan}
\examinersEmail{ozan@kth.se}
% If the examiner is from within KTH add their KTHID, School and Department info
\examinersKTHID{u1d13i2c}
\examinersSchool{\schoolAcronym{SCI}}
\examinersDepartment{Mathematics}
% other for a examiner outside of KTH add their organization info
%\examinersOrganization{Timbuktu University, Department of Pseudoscience}


% \hostcompany{Företaget AB} % Remove this line if the project was not done at a host company
%\hostorganization{CERN}   % if there was a host organization

\date{\today}

\courseCycle{2}
\courseCode{SF259X}
\courseCredits{30.0}

\programcode{COSSE}
\degreeName{Degree of Master of Science in Engineering}
% Note that the subject area for a Bachelor's thesis (Kandidatexamen) should be either Technology or Architecture
% If the thesis is in Swedish, these would be: teknik | arkitektur   -- Note the use of lower case
\subjectArea{Scientific Computing}
% if there is a second degree
%\secondProgramcode{CINTE}
%\secondDegreeName{test second degree}
%\secondSubjectArea{test second subject area}

% Note that in the case of both Both Degree of Master of Science in Engineering and Master's degree
% there are two cases: "Both" is used when the field of technology (\subjectArea{}) and the main subject (\secondSubjectArea{} are different and the case "Same" when they are the same.
%% Both case
%\courseCycle{2}
%\courseCode{xxxxxx}
%\courseCredits{30.0}
%\degreeName{Both Degree of Master of Science in Engineering and Master's degree}
%\subjectArea{Biotechnology}
%\secondSubjectArea{Medical Engineering}

%\courseCycle{2}
%\courseCode{xxxxxx}
%\courseCredits{30.0}
%\degreeName{Både civilingenjörsexamen och masterexamen}
%\subjectArea{bioteknik}
%\secondSubjectArea{medicinsk teknik}

%% Same case
%\courseCycle{2}
%\courseCode{xxxxxx}
%\courseCredits{30.0}
%\degreeName{Both Degree of Master of Science in Engineering and Master's degree}
%\subjectArea{Biotechnology}
%\secondSubjectArea{Biotechnology}

%\courseCycle{2}
%\courseCode{xxxxxx}
%\courseCredits{30.0}
%\degreeName{Både civilingenjörsexamen och masterexamen}
%\subjectArea{bioteknik}
%\secondSubjectArea{bioteknik}

% For a CDATE student the following are likely values:
%\programcode{CDATE}
%\courseCycle{2}
%\courseCode{DA231X}
%\courseCredits{30.0}
%\degreeName{Degree of Master of Science in Engineering}
%\subjectArea{Computer Science and Engineering}

% For a TCSCM student the following are likely values:
%\programcode{TCSCM}
%\courseCycle{2}
%\courseCode{DA231X}
%\courseCredits{30.0}
%\degreeName{Master's Programme, Computer Science, 120 credits}
%\subjectArea{Computer Science}

% For a CMETE student the following are likely values:
%\programcode{CMETE}
%\courseCycle{2}
%\courseCode{DA231X}
%\courseCredits{30.0}
%\degreeName{Degree of Master of Science in Engineering}
%\subjectArea{Media Technology}

% For a CINTE student the following are likely values:
%\programcode{CINTE}
%\courseCycle{2}
%\courseCode{DA231X}
%\courseCredits{30.0}
%\degreeName{Degree of Master of Science in Engineering}
%\subjectArea{Information and Communication Technology}


%%%%% for DiVA's National Subject Category information
%%% Enter one or more 3 or 5 digit codes
%%% See https://www.scb.se/contentassets/3a12f556522d4bdc887c4838a37c7ec7/standard-for-svensk-indelning--av-forskningsamnen-2011-uppdaterad-aug-2016.pdf
%%% See https://www.scb.se/contentassets/10054f2ef27c437884e8cde0d38b9cc4/oversattningsnyckel-forskningsamnen.pdf
%%%%
%%%% Some examples of these codes are shown below:
% 102 Data- och informationsvetenskap (Datateknik)    Computer and Information Sciences
% 10201 Datavetenskap (datalogi) Computer Sciences
% 10202 Systemvetenskap, informationssystem och informatik (samhällsvetenskaplig inriktning under 50804)
% Information Systems (Social aspects to be 50804)
% 10203 Bioinformatik (beräkningsbiologi) (tillämpningar under 10610)
% Bioinformatics (Computational Biology) (applications to be 10610)
% 10204 Människa-datorinteraktion (interaktionsdesign) (Samhällsvetenskapliga aspekter under 50803) Human Computer Interaction (Social aspects to be 50803)
% 10205 Programvaruteknik Software Engineering
% 10206 Datorteknik Computer Engineering
% 10207 Datorseende och robotik (autonoma system) Computer Vision and Robotics (Autonomous Systems)
% 10208 Språkteknologi (språkvetenskaplig databehandling) Language Technology (Computational Linguistics)
% 10209 Medieteknik Media and Communication Technology
% 10299 Annan data- och informationsvetenskap Other Computer and Information Science
%%%
% 202 Elektroteknik och elektronik Electrical Engineering, Electronic Engineering, Information Engineering
% 20201 Robotteknik och automation Robotics
% 20202 Reglerteknik Control Engineering
% 20203 Kommunikationssystem Communication Systems
% 20204 Telekommunikation Telecommunications
% 20205 Signalbehandling Signal Processing
% 20206 Datorsystem Computer Systems
% 20207 Inbäddad systemteknik Embedded Systems
% 20299 Annan elektroteknik och elektronik Other Electrical Engineering, Electronic Engineering, Information Engineering
%% Example for a thesis in Computer Science and Computer Systems
\nationalsubjectcategories{10201, 10206}
